\documentclass[acmlarge]{acmart}

\usepackage{booktabs} % For formal tables


\usepackage[ruled]{algorithm2e} % For algorithms
\renewcommand{\algorithmcfname}{ALGORITHM}
\SetAlFnt{\small}
\SetAlCapFnt{\small}
\SetAlCapNameFnt{\small}
\SetAlCapHSkip{0pt}
\IncMargin{-\parindent}

% Metadata Information
%\acmJournal{POMACS}
\acmVolume{9}
\acmNumber{4}
\acmArticle{39}
\acmYear{2010}
\acmMonth{3}
\acmArticleSeq{11}

%\acmBadgeR[http://ctuning.org/ae/ppopp2016.html]{ae-logo}
%\acmBadgeL[http://ctuning.org/ae/ppopp2016.html]{ae-logo}


% Copyright
%\setcopyright{acmcopyright}
%\setcopyright{acmlicensed}
%\setcopyright{rightsretained}
%\setcopyright{usgov}
\setcopyright{usgovmixed}
%\setcopyright{cagov}
%\setcopyright{cagovmixed}

% DOI
\acmDOI{0000001.0000001}

% Paper history
\received{February 2007}
\received{March 2009}
\received[accepted]{June 2009}


% Document starts
\begin{document}
% Title portion
\title{Performance Comparison of a DNA based Cryptosystem and Modern Cryptosystems} 
\author{Paul Emil Ongoco}
\orcid{1234-5678-9012-3456}
\affiliation{%
  \institution{University of the Philippines Diliman}
  \city{Quezon City}
  \country{Philippines}}
\author{Patrick Angelo Roderno}
\affiliation{%
  \institution{University of the Philippines Diliman}
  \department{School of Engineering}
  \country{Philippines}
}


\begin{abstract}
As computers become more powerful, existing cryptosystems become vulnerable to attacks. This trend caused a nonstop attempt to improve security. Since the emergence of DNA computation, researches have focused in exploring its potential applications in cryptography due to its massive parallelism and high-density information storage. Since DNA cryptography is relatively new, the existing proposals for DNA-based cryptosystems are mainly theoretical and are still to be used in real life applications. We are developing a new and improved DNA-based cryptosystem based on an existing one. Comparative study to other similar cryptosystems and standard cryptosystems (AES, etc.) will be done to determine if the new cryptosystem meet the standards of today's cryptography. Our preliminary results are the simulation of the DNA-based cryptosystem in which the new cryptosystem will be based.
\end{abstract}


%
% The code below should be generated by the tool at
% http://dl.acm.org/ccs.cfm
% Please copy and paste the code instead of the example below. 
%
\begin{CCSXML}
<ccs2012>
 <concept>
  <concept_id>10010520.10010553.10010562</concept_id>
  <concept_desc>Computer systems organization~Embedded systems</concept_desc>
  <concept_significance>500</concept_significance>
 </concept>
 <concept>
  <concept_id>10010520.10010575.10010755</concept_id>
  <concept_desc>Computer systems organization~Redundancy</concept_desc>
  <concept_significance>300</concept_significance>
 </concept>
 <concept>
  <concept_id>10010520.10010553.10010554</concept_id>
  <concept_desc>Computer systems organization~Robotics</concept_desc>
  <concept_significance>100</concept_significance>
 </concept>
 <concept>
  <concept_id>10003033.10003083.10003095</concept_id>
  <concept_desc>Networks~Network reliability</concept_desc>
  <concept_significance>100</concept_significance>
 </concept>
</ccs2012>  
\end{CCSXML}

\ccsdesc[500]{Computer systems organization~Embedded systems}
\ccsdesc[300]{Computer systems organization~Redundancy}
\ccsdesc{Computer systems organization~Robotics}
\ccsdesc[100]{Networks~Network reliability}

%
% End generated code
%


\keywords{Wireless sensor networks, media access control,
multi-channel, radio interference, time synchronization}


\thanks{This work is supported by the National Science Foundation,
  under grant CNS-0435060, grant CCR-0325197 and grant EN-CS-0329609.

  Author's addresses: G. Zhou, Computer Science Department, College of
  William and Mary; Y. Wu {and} J. A. Stankovic, Computer Science
  Department, University of Virginia; T. Yan, Eaton Innovation Center;
  T. He, Computer Science Department, University of Minnesota; C.
  Huang, Google; T. F. Abdelzaher, (Current address) NASA Ames
  Research Center, Moffett Field, California 94035.}


\maketitle

% The default list of authors is too long for headers}
\renewcommand{\shortauthors}{G. Zhou et al.}

\input{samplebody-journals}



\end{document}
